% !TEX root = ../thesis.tex

\section{引言}
    文献报告格式要求:用A4纸,通栏排版。页边距:上2.2厘米,下2厘米,左2.7厘米,右2.3厘米。必须插入居中页码。除摘要和文献清单用单倍行距外,其余(包括正文、标题、姓名、日期)一律用1.2倍行距。中文一律用宋体,英文和数字一律用Times New Roman。摘要和文献用小五号字,正文用五号,段前段后不留空行。中文摘要段前段后各空0.5行。英文摘要段前空0.5行,段后空1行。左右适当缩进。每段文字首行缩进约2个字符。

    一级标题用小四粗体,段前空0.5行,段后不留空行。二级标题用五号粗体,左端不缩进,段前后均不留多余空格。

    报告篇幅:4000~6000字,包括图表、参考文献在内7~8页,不得超过。要学会组织材料,精简文字。要求结构合理,文字通顺,格式规范,图文并茂。以消化文献内容后自己绘制的图表为优。直接拷贝的图表要严格控制数量和大小,拒绝质量太差的图。图表应与正文文字相协调。

\section{基本要求}
    主要以口头报告内容为题,也允许适当改变。要充分阅读文献,应包括近期文献和高层次文献,要有代表性,要有足够的国际文献。要列出20篇以上,注意信息完整,格式规范统一。要在正文中标明对文献的引用。

    必须独立完成,在消化文献内容、进行分析和综合基础上,用自己的话来叙述。不应详细展开某一篇文献的细节,要根据自己的理解用简明扼要的语言进行归纳。引用必要的公式,但要控制公式数量。要有自己的见解,不得进行简单拷贝。注意:以拷贝为主的报告将被认为不合格。引用的图表和主要公式应标明文献出处。

\section{注意事项}
    围绕一个主题写文献阅读报告。可根据精读文献和泛读资料,重点讲述一个研究方向。要介绍该领域的研究意义,指出相关研究热点。可扼要介绍一种代表性方法的要点,但不能通篇只涉及一两篇文献。提出一些可研究的问题,给出自己的见解。

    \subsection{文献清单}
        列出所阅读文献的清单。给出完整信息:作者,题目,期刊,卷期,页码,年份,会议名称,会议地点,网址,学位论文出处,技术报告来源,专利号等。在报告正文中标明文献引用情况。

    \subsection{一些通病}
        文献档次低、资料陈旧、数量不足、简单应付、正文中不标注文献。以转抄拼凑网上现成资料为主,囫囵吞枣,不加取舍,缺乏见解。写成介绍某一问题或方法的介绍性文章,与文献关系甚少。仅根据一两篇资料详细讲述技术细节,照本宣科,冗长乏味。Word格式粗糙马虎,形式不规范。

    \subsection{评分依据}
        阅读文献是否充分,有无近期文献,所选文献有无代表性。是否在理解文献基础上用自己的语言进行了清晰的归纳和叙述。结构是否合理,文字是否通顺,格式是否规范。有无独到见解。口头发言水平:表达是否清楚,有条理,PPT质量,讨论情况。
        
\section{报告写作提纲}
    文献阅读书面报告应包括以下内容:
    \begin{itemize}
        \item[-] 中文题目和摘要
        \item[-] 英文题目和摘要
        \item[-] 引言:讨论什么问题,有何意义,应用领域或前景,有待解决的问题等。
        \item[-] 正文:文献综述的主要内容,可根据具体情况分节。
        \item[-] 结论:归纳报告的主要内容,可提出自己的见解。
        \item[-] 参考文献:期刊论文、会议论文、网址、学位论文、书籍、专利格式如下(注意观察)。
    \end{itemize}



















