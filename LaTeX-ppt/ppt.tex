\documentclass[10pt,mathserif]{beamer}
%- 导言区开始 - 
    \setbeamertemplate{theorems}[numbered] % 增加定理编号
    \usetheme[progressbar=frametitle]{metropolis} % 主题设定
    % 颜色主题
    % \usecolortheme{seahorse}
    % \usecolortheme{beaver}
    % \usecolortheme{orchid}
    % \usecolortheme{seagull}
    % 修改ppt背景色
    \setbeamercolor{background canvas}{bg=background}
    
    \usepackage{appendixnumberbeamer}
    
    \usepackage{booktabs} % 优化表格
    \usepackage[scale=2]{ccicons}
    
    \usepackage{xspace}
    \newcommand{\themename}{\textbf{\textsc{metropolis}}\xspace}

    %- Font -
    \usepackage[slantfont,boldfont]{xeCJK} % 中文支持
        \setmainfont{Times New Roman} % 设置英文字体
        \setCJKmainfont{NSimSun} % 设置中文字体 宋体  YouYuan  NSimSun
        % \setCJKmainfont[BoldFont=NSimHei]{NSimSun}% 也可同时设置相应黑体 可以通过\bfseries命令使之变成黑体
        \newCJKfontfamily\YouYuan{YouYuan} % 设置其他中文字体
        \newCJKfontfamily\SimSun{SimSun}
        \xeCJKEditPunctStyle{quanjiao}{bound-margin-ratio = 0.25} % 符号缩进
        %\parindent = 2em   %段首缩进  %使用 indentfirst 宏包实现缩进比较好
    \usepackage{fontspec} % 字体命令
       \newfontfamily\monaco{Monaco} % 这里英语的字体命令是有用的, 中文的就不行 应该是被XeCJK覆盖了
    % \usefonttheme[onlymath]{serif} %设置数学公式应该使用衬线字体


    %- Font CMD -
    % 常用命令
    % \bfseries{} -> 加粗
    % \slshape{} -> 斜体
    % \boldsymbol{} -> 粗体字母 表示向量与矩阵
    % 字体大小自定义命令
    \newcommand{\chuhao}{\fontsize{42pt}{\baselineskip}\selectfont}
    \newcommand{\xiaochuhao}{\fontsize{36pt}{\baselineskip}\selectfont}
    \newcommand{\yihao}{\fontsize{28pt}{\baselineskip}\selectfont}
    \newcommand{\erhao}{\fontsize{21pt}{\baselineskip}\selectfont}
    \newcommand{\xiaoerhao}{\fontsize{18pt}{\baselineskip}\selectfont}
    \newcommand{\sanhao}{\fontsize{15.75pt}{\baselineskip}\selectfont}
    \newcommand{\sihao}{\fontsize{14pt}{\baselineskip}\selectfont}
    \newcommand{\xiaosihao}{\fontsize{12pt}{\baselineskip}\selectfont}
    \newcommand{\wuhao}{\fontsize{10.5pt}{\baselineskip}\selectfont}
    \newcommand{\xiaowuhao}{\fontsize{9pt}{\baselineskip}\selectfont}
    \newcommand{\liuhao}{\fontsize{7.875pt}{\baselineskip}\selectfont}
    \newcommand{\qihao}{\fontsize{5.25pt}{\baselineskip}\selectfont}
    
    %- Math -
    \usepackage{amsmath, amsthm, amssymb} % 数学支持
        % 定理环境风格
        \theoremstyle{definition}   
        %\theoremstyle{plain}
        %\theoremstyle{remark}
        % 定理 定义 性质 
        % \newtheorem{definition}{定义}[section] % 定义环境
        % \newtheorem{theorem}{定理}[section] % 定理环境
        \newtheorem{property}{性质}[section] % 性质环境
        % 证明环境
        % \newenvironment{pf}{{\noindent\bf 证明:}\quad}{\hfill $\qed$\par}  
        % \newenvironment{pf}[1]{\par \textbf{证明}\quad #1}{\hfill$\qed$ \par}
        % \renewcommand*{\qedsymbol}{[证毕]}
        % 公式编号
        \numberwithin{equation}{section} % 公式编号随章节走
    % 其他数学字体支持
    \usepackage{mathrsfs} % 字母花体
    \usepackage{dsfont} % 漂亮的R
    % 矩阵格式调整
    \makeatletter
    \def\env@matrix{\hskip -\arraycolsep
    \let\@ifnextchar\new@ifnextchar
    \linespread{1}\selectfont
    \renewcommand{\arraystretch}{1.2}%
    \array{*\c@MaxMatrixCols c}}
    \makeatother

    % 更多符号支持
    \usepackage{pifont}

    % 下划线支持
    \usepackage{ulem}
    
    %- Paragraph -
    \usepackage{indentfirst} % 设置首行缩进
        \setlength{\parindent}{2em}
    \linespread{1} % 行距
    % \setlength{\columnsep}{1em} %列间距  好像不管用
    % \setlength{\topmargin}{10mm} % 页眉到页边的距离
    % \setlength{\parskip}{10pt}  % 设置部分段落间隔
    \def\changemargin#1#2{\list{}{\rightmargin#2\leftmargin#1}\item[]}
    \let\endchangemargin=\endlist % 增加可更改段落页边距 

    %- TIKZ -
    \usepackage{tikz,pgfplots} % 画图支持    如用 matlab2tikz 的输出需要 pgfplots
        \usetikzlibrary{graphs,arrows,shapes,chains,quotes,plotmarks}
    \pgfplotsset{compat=1.16}
    \newenvironment{flowchart}{ % 一个画流程图环境
        \begin{tikzpicture}[
        %inner sep=2mm, % 边框到形状内字符距离
        > = stealth, % 箭头样式
        -> /.style={thick ,black},
        -- /.style={thick ,black},
        line width=1pt,
        every edge quotes/.style={font=\sihao,auto}, % 标签样式
        every node/.style={font=\sihao}, % 框内字体
        point/.style={inner sep=0pt,line width=0pt,fill}, % point样式
        terminal/.style={circle,draw,thin,inner sep=0pt}, % 结点样式
        rect/.style={rectangle,rounded corners=1mm,draw,thick,inner sep=5pt,minimum size=10mm,align=center}, %rect样式
        operation/.style={circle,draw,thick,inner sep=-1pt}, % operation样式
        hv path/.style={to path={-| (\tikztotarget)}}, % 直角折线
        vh path/.style={to path={|- (\tikztotarget)}},
        ]}{\end{tikzpicture}}
    
    %- Table -
    \usepackage{multirow}
    \usepackage{longtable} % 跨页表格支持 
    \renewcommand\arraystretch{1.5} % 表格行距
    \renewcommand\tabcolsep{3mm} % 表格列距
    \renewcommand\tablename{表} % 改表的标号

    %- Picture -
    \usepackage{graphicx} % 插入图片
    \usepackage{subfigure} % 插入并列图
    \usepackage{pdfpages} % 插入pdf支持
    \usepackage{caption} % 去掉表/图的标号需要
        \captionsetup[table]{labelsep=quad} % 去掉 表几 后面的冒号
        \captionsetup[figure]{labelsep=quad} % 去掉 图几 后面的冒号
        \setlength{\abovecaptionskip}{10pt} % 设定图与caption距离
        \setlength{\belowcaptionskip}{-10pt}
        \DeclareCaptionFormat{captionformat}{\wuhao #1#2#3} % 改图表标签字号
        \captionsetup{format = captionformat}
    \usepackage{wrapfig} % 字体环绕图片支持
    \usepackage{float} % 插入图片可使用 'H' 取消浮动体机制
    \usepackage{xcolor} % 颜色支持 用于代码高亮 和 tikz绘图
        \definecolor{background}{RGB}{254, 248, 230} 
    \renewcommand\figurename{图} % 改图的标号
    \usepackage{epstopdf} % 提供eps图片插入支持    matlab2tikz 的实现并不完美 还是插入eps比较好用
    %% ppt插入eps格式图片注意不能是这一页ppt的第一个元素, 否则图片坐标轴会糊, 不知到为什么
    
    %- Code -
    \usepackage{listings} % 代码环境
        \lstset{
            showstringspaces=false,
            columns=fixed,
            numbers=left,
            frame=none,
            backgroundcolor=\color[RGB]{245,245,220},
            keywordstyle=\color[RGB]{0,0,205},
            %numberstyle=\footnotesize\color{darkgray},
            commentstyle=\it\color[RGB]{0,96,96},
            stringstyle=\rmfamily\slshape\monaco\color[RGB]{47,79,79},
            numberstyle=\tiny\monaco,
            basicstyle=\tiny\monaco,
            escapeinside=``,
            language=c++,
        }

        %- Bibliography -
        % 参考文献角标设定
        % 角标设定为: 数字编号、排序且压缩、上标、外侧方括    压缩是说当引用多与三个时使用 '-' 连接
        \usepackage[numbers, sort&compress, super, square]{natbib} 
        \renewcommand{\refname}{\bfseries\wuhao{参考文献}} % article使用这个
        % \renewcommand{\bibname}{参考文献} % book使用这个
        \renewcommand{\bibfont}{\wuhao} % 参考文献项的字体
        \renewcommand{\bibnumfmt}[1]{\YouYuan{[#1]}} % 修改文献编号样式
            \setlength{\bibsep}{0.25ex} % 文献条目之间的距离    单位 倍行距
        % \bibliographystyle{unsrt} % 设置参考文献排序的类型 按引用顺序 用gbt7714就不需要这个了
        \usepackage{gbt7714} % 国标文献格式支持

    % 附录
    \usepackage{appendix}

    % \usepackage{eulervm}

    % 引用
    % \usepackage{showkeys} % 提示引用label 

    %- Link -
    % \usepackage[hidelinks]{hyperref} % 超链接 与 书签  提供\phantomsection

    % ppt 封面信息
    \title{高斯过程回归方法及其应用}
    \subtitle{Gaussian processes regression and its application}
    \date{XXX}
    \author{XX}
    % \institute{ $\cdot$}
    % \titlegraphic{\hfill\includegraphics[height=1.5cm]{logo.pdf}}

%- 导言区结束 -

\begin{document}

    \maketitle % 显示封面

    \documentclass[10pt,mathserif]{beamer}
%- 导言区开始 - 
    \setbeamertemplate{theorems}[numbered] % 增加定理编号
    \usetheme[progressbar=frametitle]{metropolis} % 主题设定
    % 颜色主题
    % \usecolortheme{seahorse}
    % \usecolortheme{beaver}
    % \usecolortheme{orchid}
    % \usecolortheme{seagull}
    % 修改ppt背景色
    \setbeamercolor{background canvas}{bg=background}
    
    \usepackage{appendixnumberbeamer}
    
    \usepackage{booktabs} % 优化表格
    \usepackage[scale=2]{ccicons}
    
    \usepackage{xspace}
    \newcommand{\themename}{\textbf{\textsc{metropolis}}\xspace}

    %- Font -
    \usepackage[slantfont,boldfont]{xeCJK} % 中文支持
        \setmainfont{Times New Roman} % 设置英文字体
        \setCJKmainfont{NSimSun} % 设置中文字体 宋体  YouYuan  NSimSun
        % \setCJKmainfont[BoldFont=NSimHei]{NSimSun}% 也可同时设置相应黑体 可以通过\bfseries命令使之变成黑体
        \newCJKfontfamily\YouYuan{YouYuan} % 设置其他中文字体
        \newCJKfontfamily\SimSun{SimSun}
        \xeCJKEditPunctStyle{quanjiao}{bound-margin-ratio = 0.25} % 符号缩进
        %\parindent = 2em   %段首缩进  %使用 indentfirst 宏包实现缩进比较好
    \usepackage{fontspec} % 字体命令
       \newfontfamily\monaco{Monaco} % 这里英语的字体命令是有用的, 中文的就不行 应该是被XeCJK覆盖了
    % \usefonttheme[onlymath]{serif} %设置数学公式应该使用衬线字体


    %- Font CMD -
    % 常用命令
    % \bfseries{} -> 加粗
    % \slshape{} -> 斜体
    % \boldsymbol{} -> 粗体字母 表示向量与矩阵
    % 字体大小自定义命令
    \newcommand{\chuhao}{\fontsize{42pt}{\baselineskip}\selectfont}
    \newcommand{\xiaochuhao}{\fontsize{36pt}{\baselineskip}\selectfont}
    \newcommand{\yihao}{\fontsize{28pt}{\baselineskip}\selectfont}
    \newcommand{\erhao}{\fontsize{21pt}{\baselineskip}\selectfont}
    \newcommand{\xiaoerhao}{\fontsize{18pt}{\baselineskip}\selectfont}
    \newcommand{\sanhao}{\fontsize{15.75pt}{\baselineskip}\selectfont}
    \newcommand{\sihao}{\fontsize{14pt}{\baselineskip}\selectfont}
    \newcommand{\xiaosihao}{\fontsize{12pt}{\baselineskip}\selectfont}
    \newcommand{\wuhao}{\fontsize{10.5pt}{\baselineskip}\selectfont}
    \newcommand{\xiaowuhao}{\fontsize{9pt}{\baselineskip}\selectfont}
    \newcommand{\liuhao}{\fontsize{7.875pt}{\baselineskip}\selectfont}
    \newcommand{\qihao}{\fontsize{5.25pt}{\baselineskip}\selectfont}
    
    %- Math -
    \usepackage{amsmath, amsthm, amssymb} % 数学支持
        % 定理环境风格
        \theoremstyle{definition}   
        %\theoremstyle{plain}
        %\theoremstyle{remark}
        % 定理 定义 性质 
        % \newtheorem{definition}{定义}[section] % 定义环境
        % \newtheorem{theorem}{定理}[section] % 定理环境
        \newtheorem{property}{性质}[section] % 性质环境
        % 证明环境
        % \newenvironment{pf}{{\noindent\bf 证明:}\quad}{\hfill $\qed$\par}  
        % \newenvironment{pf}[1]{\par \textbf{证明}\quad #1}{\hfill$\qed$ \par}
        % \renewcommand*{\qedsymbol}{[证毕]}
        % 公式编号
        \numberwithin{equation}{section} % 公式编号随章节走
    % 其他数学字体支持
    \usepackage{mathrsfs} % 字母花体
    \usepackage{dsfont} % 漂亮的R
    % 矩阵格式调整
    \makeatletter
    \def\env@matrix{\hskip -\arraycolsep
    \let\@ifnextchar\new@ifnextchar
    \linespread{1}\selectfont
    \renewcommand{\arraystretch}{1.2}%
    \array{*\c@MaxMatrixCols c}}
    \makeatother

    % 更多符号支持
    \usepackage{pifont}

    % 下划线支持
    \usepackage{ulem}
    
    %- Paragraph -
    \usepackage{indentfirst} % 设置首行缩进
        \setlength{\parindent}{2em}
    \linespread{1} % 行距
    % \setlength{\columnsep}{1em} %列间距  好像不管用
    % \setlength{\topmargin}{10mm} % 页眉到页边的距离
    % \setlength{\parskip}{10pt}  % 设置部分段落间隔
    \def\changemargin#1#2{\list{}{\rightmargin#2\leftmargin#1}\item[]}
    \let\endchangemargin=\endlist % 增加可更改段落页边距 

    %- TIKZ -
    \usepackage{tikz,pgfplots} % 画图支持    如用 matlab2tikz 的输出需要 pgfplots
        \usetikzlibrary{graphs,arrows,shapes,chains,quotes,plotmarks}
    \pgfplotsset{compat=1.16}
    \newenvironment{flowchart}{ % 一个画流程图环境
        \begin{tikzpicture}[
        %inner sep=2mm, % 边框到形状内字符距离
        > = stealth, % 箭头样式
        -> /.style={thick ,black},
        -- /.style={thick ,black},
        line width=1pt,
        every edge quotes/.style={font=\sihao,auto}, % 标签样式
        every node/.style={font=\sihao}, % 框内字体
        point/.style={inner sep=0pt,line width=0pt,fill}, % point样式
        terminal/.style={circle,draw,thin,inner sep=0pt}, % 结点样式
        rect/.style={rectangle,rounded corners=1mm,draw,thick,inner sep=5pt,minimum size=10mm,align=center}, %rect样式
        operation/.style={circle,draw,thick,inner sep=-1pt}, % operation样式
        hv path/.style={to path={-| (\tikztotarget)}}, % 直角折线
        vh path/.style={to path={|- (\tikztotarget)}},
        ]}{\end{tikzpicture}}
    
    %- Table -
    \usepackage{multirow}
    \usepackage{longtable} % 跨页表格支持 
    \renewcommand\arraystretch{1.5} % 表格行距
    \renewcommand\tabcolsep{3mm} % 表格列距
    \renewcommand\tablename{表} % 改表的标号

    %- Picture -
    \usepackage{graphicx} % 插入图片
    \usepackage{subfigure} % 插入并列图
    \usepackage{pdfpages} % 插入pdf支持
    \usepackage{caption} % 去掉表/图的标号需要
        \captionsetup[table]{labelsep=quad} % 去掉 表几 后面的冒号
        \captionsetup[figure]{labelsep=quad} % 去掉 图几 后面的冒号
        \setlength{\abovecaptionskip}{10pt} % 设定图与caption距离
        \setlength{\belowcaptionskip}{-10pt}
        \DeclareCaptionFormat{captionformat}{\wuhao #1#2#3} % 改图表标签字号
        \captionsetup{format = captionformat}
    \usepackage{wrapfig} % 字体环绕图片支持
    \usepackage{float} % 插入图片可使用 'H' 取消浮动体机制
    \usepackage{xcolor} % 颜色支持 用于代码高亮 和 tikz绘图
        \definecolor{background}{RGB}{254, 248, 230} 
    \renewcommand\figurename{图} % 改图的标号
    \usepackage{epstopdf} % 提供eps图片插入支持    matlab2tikz 的实现并不完美 还是插入eps比较好用
    %% ppt插入eps格式图片注意不能是这一页ppt的第一个元素, 否则图片坐标轴会糊, 不知到为什么
    
    %- Code -
    \usepackage{listings} % 代码环境
        \lstset{
            showstringspaces=false,
            columns=fixed,
            numbers=left,
            frame=none,
            backgroundcolor=\color[RGB]{245,245,220},
            keywordstyle=\color[RGB]{0,0,205},
            %numberstyle=\footnotesize\color{darkgray},
            commentstyle=\it\color[RGB]{0,96,96},
            stringstyle=\rmfamily\slshape\monaco\color[RGB]{47,79,79},
            numberstyle=\tiny\monaco,
            basicstyle=\tiny\monaco,
            escapeinside=``,
            language=c++,
        }

        %- Bibliography -
        % 参考文献角标设定
        % 角标设定为: 数字编号、排序且压缩、上标、外侧方括    压缩是说当引用多与三个时使用 '-' 连接
        \usepackage[numbers, sort&compress, super, square]{natbib} 
        \renewcommand{\refname}{\bfseries\wuhao{参考文献}} % article使用这个
        % \renewcommand{\bibname}{参考文献} % book使用这个
        \renewcommand{\bibfont}{\wuhao} % 参考文献项的字体
        \renewcommand{\bibnumfmt}[1]{\YouYuan{[#1]}} % 修改文献编号样式
            \setlength{\bibsep}{0.25ex} % 文献条目之间的距离    单位 倍行距
        % \bibliographystyle{unsrt} % 设置参考文献排序的类型 按引用顺序 用gbt7714就不需要这个了
        \usepackage{gbt7714} % 国标文献格式支持

    % 附录
    \usepackage{appendix}

    % \usepackage{eulervm}

    % 引用
    % \usepackage{showkeys} % 提示引用label 

    %- Link -
    % \usepackage[hidelinks]{hyperref} % 超链接 与 书签  提供\phantomsection

    % ppt 封面信息
    \title{高斯过程回归方法及其应用}
    \subtitle{Gaussian processes regression and its application}
    \date{XXX}
    \author{XX}
    % \institute{ $\cdot$}
    % \titlegraphic{\hfill\includegraphics[height=1.5cm]{logo.pdf}}

%- 导言区结束 -

\begin{document}

    \maketitle % 显示封面

    \documentclass[10pt,mathserif]{beamer}
%- 导言区开始 - 
    \setbeamertemplate{theorems}[numbered] % 增加定理编号
    \usetheme[progressbar=frametitle]{metropolis} % 主题设定
    % 颜色主题
    % \usecolortheme{seahorse}
    % \usecolortheme{beaver}
    % \usecolortheme{orchid}
    % \usecolortheme{seagull}
    % 修改ppt背景色
    \setbeamercolor{background canvas}{bg=background}
    
    \usepackage{appendixnumberbeamer}
    
    \usepackage{booktabs} % 优化表格
    \usepackage[scale=2]{ccicons}
    
    \usepackage{xspace}
    \newcommand{\themename}{\textbf{\textsc{metropolis}}\xspace}

    %- Font -
    \usepackage[slantfont,boldfont]{xeCJK} % 中文支持
        \setmainfont{Times New Roman} % 设置英文字体
        \setCJKmainfont{NSimSun} % 设置中文字体 宋体  YouYuan  NSimSun
        % \setCJKmainfont[BoldFont=NSimHei]{NSimSun}% 也可同时设置相应黑体 可以通过\bfseries命令使之变成黑体
        \newCJKfontfamily\YouYuan{YouYuan} % 设置其他中文字体
        \newCJKfontfamily\SimSun{SimSun}
        \xeCJKEditPunctStyle{quanjiao}{bound-margin-ratio = 0.25} % 符号缩进
        %\parindent = 2em   %段首缩进  %使用 indentfirst 宏包实现缩进比较好
    \usepackage{fontspec} % 字体命令
       \newfontfamily\monaco{Monaco} % 这里英语的字体命令是有用的, 中文的就不行 应该是被XeCJK覆盖了
    % \usefonttheme[onlymath]{serif} %设置数学公式应该使用衬线字体


    %- Font CMD -
    % 常用命令
    % \bfseries{} -> 加粗
    % \slshape{} -> 斜体
    % \boldsymbol{} -> 粗体字母 表示向量与矩阵
    % 字体大小自定义命令
    \newcommand{\chuhao}{\fontsize{42pt}{\baselineskip}\selectfont}
    \newcommand{\xiaochuhao}{\fontsize{36pt}{\baselineskip}\selectfont}
    \newcommand{\yihao}{\fontsize{28pt}{\baselineskip}\selectfont}
    \newcommand{\erhao}{\fontsize{21pt}{\baselineskip}\selectfont}
    \newcommand{\xiaoerhao}{\fontsize{18pt}{\baselineskip}\selectfont}
    \newcommand{\sanhao}{\fontsize{15.75pt}{\baselineskip}\selectfont}
    \newcommand{\sihao}{\fontsize{14pt}{\baselineskip}\selectfont}
    \newcommand{\xiaosihao}{\fontsize{12pt}{\baselineskip}\selectfont}
    \newcommand{\wuhao}{\fontsize{10.5pt}{\baselineskip}\selectfont}
    \newcommand{\xiaowuhao}{\fontsize{9pt}{\baselineskip}\selectfont}
    \newcommand{\liuhao}{\fontsize{7.875pt}{\baselineskip}\selectfont}
    \newcommand{\qihao}{\fontsize{5.25pt}{\baselineskip}\selectfont}
    
    %- Math -
    \usepackage{amsmath, amsthm, amssymb} % 数学支持
        % 定理环境风格
        \theoremstyle{definition}   
        %\theoremstyle{plain}
        %\theoremstyle{remark}
        % 定理 定义 性质 
        % \newtheorem{definition}{定义}[section] % 定义环境
        % \newtheorem{theorem}{定理}[section] % 定理环境
        \newtheorem{property}{性质}[section] % 性质环境
        % 证明环境
        % \newenvironment{pf}{{\noindent\bf 证明:}\quad}{\hfill $\qed$\par}  
        % \newenvironment{pf}[1]{\par \textbf{证明}\quad #1}{\hfill$\qed$ \par}
        % \renewcommand*{\qedsymbol}{[证毕]}
        % 公式编号
        \numberwithin{equation}{section} % 公式编号随章节走
    % 其他数学字体支持
    \usepackage{mathrsfs} % 字母花体
    \usepackage{dsfont} % 漂亮的R
    % 矩阵格式调整
    \makeatletter
    \def\env@matrix{\hskip -\arraycolsep
    \let\@ifnextchar\new@ifnextchar
    \linespread{1}\selectfont
    \renewcommand{\arraystretch}{1.2}%
    \array{*\c@MaxMatrixCols c}}
    \makeatother

    % 更多符号支持
    \usepackage{pifont}

    % 下划线支持
    \usepackage{ulem}
    
    %- Paragraph -
    \usepackage{indentfirst} % 设置首行缩进
        \setlength{\parindent}{2em}
    \linespread{1} % 行距
    % \setlength{\columnsep}{1em} %列间距  好像不管用
    % \setlength{\topmargin}{10mm} % 页眉到页边的距离
    % \setlength{\parskip}{10pt}  % 设置部分段落间隔
    \def\changemargin#1#2{\list{}{\rightmargin#2\leftmargin#1}\item[]}
    \let\endchangemargin=\endlist % 增加可更改段落页边距 

    %- TIKZ -
    \usepackage{tikz,pgfplots} % 画图支持    如用 matlab2tikz 的输出需要 pgfplots
        \usetikzlibrary{graphs,arrows,shapes,chains,quotes,plotmarks}
    \pgfplotsset{compat=1.16}
    \newenvironment{flowchart}{ % 一个画流程图环境
        \begin{tikzpicture}[
        %inner sep=2mm, % 边框到形状内字符距离
        > = stealth, % 箭头样式
        -> /.style={thick ,black},
        -- /.style={thick ,black},
        line width=1pt,
        every edge quotes/.style={font=\sihao,auto}, % 标签样式
        every node/.style={font=\sihao}, % 框内字体
        point/.style={inner sep=0pt,line width=0pt,fill}, % point样式
        terminal/.style={circle,draw,thin,inner sep=0pt}, % 结点样式
        rect/.style={rectangle,rounded corners=1mm,draw,thick,inner sep=5pt,minimum size=10mm,align=center}, %rect样式
        operation/.style={circle,draw,thick,inner sep=-1pt}, % operation样式
        hv path/.style={to path={-| (\tikztotarget)}}, % 直角折线
        vh path/.style={to path={|- (\tikztotarget)}},
        ]}{\end{tikzpicture}}
    
    %- Table -
    \usepackage{multirow}
    \usepackage{longtable} % 跨页表格支持 
    \renewcommand\arraystretch{1.5} % 表格行距
    \renewcommand\tabcolsep{3mm} % 表格列距
    \renewcommand\tablename{表} % 改表的标号

    %- Picture -
    \usepackage{graphicx} % 插入图片
    \usepackage{subfigure} % 插入并列图
    \usepackage{pdfpages} % 插入pdf支持
    \usepackage{caption} % 去掉表/图的标号需要
        \captionsetup[table]{labelsep=quad} % 去掉 表几 后面的冒号
        \captionsetup[figure]{labelsep=quad} % 去掉 图几 后面的冒号
        \setlength{\abovecaptionskip}{10pt} % 设定图与caption距离
        \setlength{\belowcaptionskip}{-10pt}
        \DeclareCaptionFormat{captionformat}{\wuhao #1#2#3} % 改图表标签字号
        \captionsetup{format = captionformat}
    \usepackage{wrapfig} % 字体环绕图片支持
    \usepackage{float} % 插入图片可使用 'H' 取消浮动体机制
    \usepackage{xcolor} % 颜色支持 用于代码高亮 和 tikz绘图
        \definecolor{background}{RGB}{254, 248, 230} 
    \renewcommand\figurename{图} % 改图的标号
    \usepackage{epstopdf} % 提供eps图片插入支持    matlab2tikz 的实现并不完美 还是插入eps比较好用
    %% ppt插入eps格式图片注意不能是这一页ppt的第一个元素, 否则图片坐标轴会糊, 不知到为什么
    
    %- Code -
    \usepackage{listings} % 代码环境
        \lstset{
            showstringspaces=false,
            columns=fixed,
            numbers=left,
            frame=none,
            backgroundcolor=\color[RGB]{245,245,220},
            keywordstyle=\color[RGB]{0,0,205},
            %numberstyle=\footnotesize\color{darkgray},
            commentstyle=\it\color[RGB]{0,96,96},
            stringstyle=\rmfamily\slshape\monaco\color[RGB]{47,79,79},
            numberstyle=\tiny\monaco,
            basicstyle=\tiny\monaco,
            escapeinside=``,
            language=c++,
        }

        %- Bibliography -
        % 参考文献角标设定
        % 角标设定为: 数字编号、排序且压缩、上标、外侧方括    压缩是说当引用多与三个时使用 '-' 连接
        \usepackage[numbers, sort&compress, super, square]{natbib} 
        \renewcommand{\refname}{\bfseries\wuhao{参考文献}} % article使用这个
        % \renewcommand{\bibname}{参考文献} % book使用这个
        \renewcommand{\bibfont}{\wuhao} % 参考文献项的字体
        \renewcommand{\bibnumfmt}[1]{\YouYuan{[#1]}} % 修改文献编号样式
            \setlength{\bibsep}{0.25ex} % 文献条目之间的距离    单位 倍行距
        % \bibliographystyle{unsrt} % 设置参考文献排序的类型 按引用顺序 用gbt7714就不需要这个了
        \usepackage{gbt7714} % 国标文献格式支持

    % 附录
    \usepackage{appendix}

    % \usepackage{eulervm}

    % 引用
    % \usepackage{showkeys} % 提示引用label 

    %- Link -
    % \usepackage[hidelinks]{hyperref} % 超链接 与 书签  提供\phantomsection

    % ppt 封面信息
    \title{高斯过程回归方法及其应用}
    \subtitle{Gaussian processes regression and its application}
    \date{XXX}
    \author{XX}
    % \institute{ $\cdot$}
    % \titlegraphic{\hfill\includegraphics[height=1.5cm]{logo.pdf}}

%- 导言区结束 -

\begin{document}

    \maketitle % 显示封面

    \documentclass[10pt,mathserif]{beamer}
%- 导言区开始 - 
    \setbeamertemplate{theorems}[numbered] % 增加定理编号
    \usetheme[progressbar=frametitle]{metropolis} % 主题设定
    % 颜色主题
    % \usecolortheme{seahorse}
    % \usecolortheme{beaver}
    % \usecolortheme{orchid}
    % \usecolortheme{seagull}
    % 修改ppt背景色
    \setbeamercolor{background canvas}{bg=background}
    
    \usepackage{appendixnumberbeamer}
    
    \usepackage{booktabs} % 优化表格
    \usepackage[scale=2]{ccicons}
    
    \usepackage{xspace}
    \newcommand{\themename}{\textbf{\textsc{metropolis}}\xspace}

    %- Font -
    \usepackage[slantfont,boldfont]{xeCJK} % 中文支持
        \setmainfont{Times New Roman} % 设置英文字体
        \setCJKmainfont{NSimSun} % 设置中文字体 宋体  YouYuan  NSimSun
        % \setCJKmainfont[BoldFont=NSimHei]{NSimSun}% 也可同时设置相应黑体 可以通过\bfseries命令使之变成黑体
        \newCJKfontfamily\YouYuan{YouYuan} % 设置其他中文字体
        \newCJKfontfamily\SimSun{SimSun}
        \xeCJKEditPunctStyle{quanjiao}{bound-margin-ratio = 0.25} % 符号缩进
        %\parindent = 2em   %段首缩进  %使用 indentfirst 宏包实现缩进比较好
    \usepackage{fontspec} % 字体命令
       \newfontfamily\monaco{Monaco} % 这里英语的字体命令是有用的, 中文的就不行 应该是被XeCJK覆盖了
    % \usefonttheme[onlymath]{serif} %设置数学公式应该使用衬线字体


    %- Font CMD -
    % 常用命令
    % \bfseries{} -> 加粗
    % \slshape{} -> 斜体
    % \boldsymbol{} -> 粗体字母 表示向量与矩阵
    % 字体大小自定义命令
    \newcommand{\chuhao}{\fontsize{42pt}{\baselineskip}\selectfont}
    \newcommand{\xiaochuhao}{\fontsize{36pt}{\baselineskip}\selectfont}
    \newcommand{\yihao}{\fontsize{28pt}{\baselineskip}\selectfont}
    \newcommand{\erhao}{\fontsize{21pt}{\baselineskip}\selectfont}
    \newcommand{\xiaoerhao}{\fontsize{18pt}{\baselineskip}\selectfont}
    \newcommand{\sanhao}{\fontsize{15.75pt}{\baselineskip}\selectfont}
    \newcommand{\sihao}{\fontsize{14pt}{\baselineskip}\selectfont}
    \newcommand{\xiaosihao}{\fontsize{12pt}{\baselineskip}\selectfont}
    \newcommand{\wuhao}{\fontsize{10.5pt}{\baselineskip}\selectfont}
    \newcommand{\xiaowuhao}{\fontsize{9pt}{\baselineskip}\selectfont}
    \newcommand{\liuhao}{\fontsize{7.875pt}{\baselineskip}\selectfont}
    \newcommand{\qihao}{\fontsize{5.25pt}{\baselineskip}\selectfont}
    
    %- Math -
    \usepackage{amsmath, amsthm, amssymb} % 数学支持
        % 定理环境风格
        \theoremstyle{definition}   
        %\theoremstyle{plain}
        %\theoremstyle{remark}
        % 定理 定义 性质 
        % \newtheorem{definition}{定义}[section] % 定义环境
        % \newtheorem{theorem}{定理}[section] % 定理环境
        \newtheorem{property}{性质}[section] % 性质环境
        % 证明环境
        % \newenvironment{pf}{{\noindent\bf 证明:}\quad}{\hfill $\qed$\par}  
        % \newenvironment{pf}[1]{\par \textbf{证明}\quad #1}{\hfill$\qed$ \par}
        % \renewcommand*{\qedsymbol}{[证毕]}
        % 公式编号
        \numberwithin{equation}{section} % 公式编号随章节走
    % 其他数学字体支持
    \usepackage{mathrsfs} % 字母花体
    \usepackage{dsfont} % 漂亮的R
    % 矩阵格式调整
    \makeatletter
    \def\env@matrix{\hskip -\arraycolsep
    \let\@ifnextchar\new@ifnextchar
    \linespread{1}\selectfont
    \renewcommand{\arraystretch}{1.2}%
    \array{*\c@MaxMatrixCols c}}
    \makeatother

    % 更多符号支持
    \usepackage{pifont}

    % 下划线支持
    \usepackage{ulem}
    
    %- Paragraph -
    \usepackage{indentfirst} % 设置首行缩进
        \setlength{\parindent}{2em}
    \linespread{1} % 行距
    % \setlength{\columnsep}{1em} %列间距  好像不管用
    % \setlength{\topmargin}{10mm} % 页眉到页边的距离
    % \setlength{\parskip}{10pt}  % 设置部分段落间隔
    \def\changemargin#1#2{\list{}{\rightmargin#2\leftmargin#1}\item[]}
    \let\endchangemargin=\endlist % 增加可更改段落页边距 

    %- TIKZ -
    \usepackage{tikz,pgfplots} % 画图支持    如用 matlab2tikz 的输出需要 pgfplots
        \usetikzlibrary{graphs,arrows,shapes,chains,quotes,plotmarks}
    \pgfplotsset{compat=1.16}
    \newenvironment{flowchart}{ % 一个画流程图环境
        \begin{tikzpicture}[
        %inner sep=2mm, % 边框到形状内字符距离
        > = stealth, % 箭头样式
        -> /.style={thick ,black},
        -- /.style={thick ,black},
        line width=1pt,
        every edge quotes/.style={font=\sihao,auto}, % 标签样式
        every node/.style={font=\sihao}, % 框内字体
        point/.style={inner sep=0pt,line width=0pt,fill}, % point样式
        terminal/.style={circle,draw,thin,inner sep=0pt}, % 结点样式
        rect/.style={rectangle,rounded corners=1mm,draw,thick,inner sep=5pt,minimum size=10mm,align=center}, %rect样式
        operation/.style={circle,draw,thick,inner sep=-1pt}, % operation样式
        hv path/.style={to path={-| (\tikztotarget)}}, % 直角折线
        vh path/.style={to path={|- (\tikztotarget)}},
        ]}{\end{tikzpicture}}
    
    %- Table -
    \usepackage{multirow}
    \usepackage{longtable} % 跨页表格支持 
    \renewcommand\arraystretch{1.5} % 表格行距
    \renewcommand\tabcolsep{3mm} % 表格列距
    \renewcommand\tablename{表} % 改表的标号

    %- Picture -
    \usepackage{graphicx} % 插入图片
    \usepackage{subfigure} % 插入并列图
    \usepackage{pdfpages} % 插入pdf支持
    \usepackage{caption} % 去掉表/图的标号需要
        \captionsetup[table]{labelsep=quad} % 去掉 表几 后面的冒号
        \captionsetup[figure]{labelsep=quad} % 去掉 图几 后面的冒号
        \setlength{\abovecaptionskip}{10pt} % 设定图与caption距离
        \setlength{\belowcaptionskip}{-10pt}
        \DeclareCaptionFormat{captionformat}{\wuhao #1#2#3} % 改图表标签字号
        \captionsetup{format = captionformat}
    \usepackage{wrapfig} % 字体环绕图片支持
    \usepackage{float} % 插入图片可使用 'H' 取消浮动体机制
    \usepackage{xcolor} % 颜色支持 用于代码高亮 和 tikz绘图
        \definecolor{background}{RGB}{254, 248, 230} 
    \renewcommand\figurename{图} % 改图的标号
    \usepackage{epstopdf} % 提供eps图片插入支持    matlab2tikz 的实现并不完美 还是插入eps比较好用
    %% ppt插入eps格式图片注意不能是这一页ppt的第一个元素, 否则图片坐标轴会糊, 不知到为什么
    
    %- Code -
    \usepackage{listings} % 代码环境
        \lstset{
            showstringspaces=false,
            columns=fixed,
            numbers=left,
            frame=none,
            backgroundcolor=\color[RGB]{245,245,220},
            keywordstyle=\color[RGB]{0,0,205},
            %numberstyle=\footnotesize\color{darkgray},
            commentstyle=\it\color[RGB]{0,96,96},
            stringstyle=\rmfamily\slshape\monaco\color[RGB]{47,79,79},
            numberstyle=\tiny\monaco,
            basicstyle=\tiny\monaco,
            escapeinside=``,
            language=c++,
        }

        %- Bibliography -
        % 参考文献角标设定
        % 角标设定为: 数字编号、排序且压缩、上标、外侧方括    压缩是说当引用多与三个时使用 '-' 连接
        \usepackage[numbers, sort&compress, super, square]{natbib} 
        \renewcommand{\refname}{\bfseries\wuhao{参考文献}} % article使用这个
        % \renewcommand{\bibname}{参考文献} % book使用这个
        \renewcommand{\bibfont}{\wuhao} % 参考文献项的字体
        \renewcommand{\bibnumfmt}[1]{\YouYuan{[#1]}} % 修改文献编号样式
            \setlength{\bibsep}{0.25ex} % 文献条目之间的距离    单位 倍行距
        % \bibliographystyle{unsrt} % 设置参考文献排序的类型 按引用顺序 用gbt7714就不需要这个了
        \usepackage{gbt7714} % 国标文献格式支持

    % 附录
    \usepackage{appendix}

    % \usepackage{eulervm}

    % 引用
    % \usepackage{showkeys} % 提示引用label 

    %- Link -
    % \usepackage[hidelinks]{hyperref} % 超链接 与 书签  提供\phantomsection

    % ppt 封面信息
    \title{高斯过程回归方法及其应用}
    \subtitle{Gaussian processes regression and its application}
    \date{XXX}
    \author{XX}
    % \institute{ $\cdot$}
    % \titlegraphic{\hfill\includegraphics[height=1.5cm]{logo.pdf}}

%- 导言区结束 -

\begin{document}

    \maketitle % 显示封面

    \input{./ppt/ppt.tex}



\end{document}
% end



\end{document}
% end



\end{document}
% end



\end{document}
% end